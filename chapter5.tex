\chapter{Fazit}

Das Ziel dieser Arbeit war zu zeigen, dass vermeintlich abstrakte und theoretisch aussehende mathematische Konzepte weitreichende Anwendung in der Praxis finden.
Zu diesem Zweck wurde die Singulärwertzerlegung als eine vielseitige Methode der linearen Algebra vorgestellt.
Obwohl sie in ihrer Beschreibung simpel erscheint --- die Zerlegung einer beliebigen Matrix in eine Drehung, Skalierung und erneute Drehung ---, besitzt sie Eigenschaften, die in diversen Anwendungsgebieten von großem Nutzen sind.
Besonders hervorzuheben ist die Aussage des Eckart-Young-Satzes, wonach die trunkierte SVD die beste Rank-\(k\)-Approximation einer Matrix bietet. 
Zusammen mit ihrer numerischen Stabilität macht sie das zu einem wichtigen Werkzeug, insbesondere im aktuellen Kontext von künstlicher Intelligenz und Big Data, wo umfangreiche Datenmengen effizient verarbeitet und reduziert werden müssen.

Die in dieser Arbeit diskutierten Anwendungen --- die Hauptkomponentenanalyse und Empfehlungssysteme --- verdeutlichen bereits die praktische Relevanz der SVD. % chktex 13
Weiterführend wäre es allerdings interessant, sie mit Methoden anderer Disziplinen zu verbinden, um komplexere Themenfelder kennenzulernen.
Anbieten dafür würde sich eine Kombination mit \emph{Deep Learning} (einem Teilbereich des maschinellen Lernens), wie beispielsweise in~\cite{diaz-moralesDeepLearningCombined2024} beschrieben wird.

Wir haben uns in dieser Arbeit ausschließlich mit reellen Matrizen beschäftigt.
Jedoch existiert die Singulärwertzerlegung auch für Matrizen mit komplexen Elementen, wobei die zugrundeliegende Theorie weitgehend analog bleibt.
Dies eröffnet zusätzliche Möglichkeiten für diverse Verwendungsgebiete, insbesondere in Bereichen wie der Quantenmechanik~\cite{martinExtraordinarySVD2012} oder der Kommunikationstechnik~\cite{tseFundamentalsWirelessCommunication2005}.
Ein vertiefender Blick auf die Anwendung der SVD in diesen Kontexten könnte eine spannende Richtung für zukünftige Untersuchungen darstellen.

