\setchapterpreamble[u]{%
    \dictum[Gilbert Strang]{The Singular Value Decomposition is a highlight of linear algebra.\footnotemark}%
    \bigskip%
}
\chapter{Einleitung}
\footnotetext{Zitiert nach~\cite[363]{strangIntroductionLinearAlgebra2009}.}
In den Jahren \num{2006} bis \num{2009} zog ein Wettbewerb, ausgerufen von der Streaming-Plattform Netflix, große mediale Aufmerksamkeit auf sich.
Es wurde ein Preisgeld von einer Million US-Dollar ausgerufen für das Team, welches den vorhanden Empfehlungsalgorithmus von Netflix zur Vorhersage von Nutzerbewertungen signifikant verbessern konnte.
Die vielversprechendsten Ansätze basierten allesamt auf dem Prinzip der \emph{Matrixfaktorisierung}, also der Zerlegung einer Matrix in mehrere andere Matrizen~\cite{korenMatrixFactorizationTechniques2009}.
Eine besondere Art der Matrixfaktorisierung bildet die Singulärwertzerlegung (engl.\ \emph{Singular Value Decomposition}, SVD), da sie im Gegensatz zu vielen anderen Matrixzerlegungen auf jede beliebige Matrix angewendet werden kann und dabei ebenso spezielle wie nützliche Eigenschaften mit sich bringt.
Dank dieser Eigenschaften bildete sie eine zentrale Grundlage der leistungsfähigsten Modelle innerhalb des Netflix-Wettbewerbs.

Der Verwendungszweck der Singulärwertzerlegung beschränkt sich allerdings nicht nur auf Empfehlungssysteme.
Ein weiteres populäres Anwendungsgebiet findet sich in der \emph{Data Science}, einem rapide wachsenden Bereich, in dem große Datenmengen sinnvoll verarbeitet werden müssen.
Andere Einsatzmöglichkeiten sind die Bildverarbeitung, Molekularbiologie, Robotik oder sogar hochspezielle Themen wie die Kristallisierungsraten von Gesteinen~\cite{martinExtraordinarySVD2012}.

Das Ziel dieser Arbeit ist herauszuarbeiten, was die Singulärwertzerlegung so besonders macht und wie dieses zunächst sehr theoretisch wirkende Konzept Anwendung in der Praxis findet.
Dafür werden zunächst die wesentlichen mathematischen Grundlagen gelegt und die oben erwähnten speziellen Eigenschaften der SVD hergeleitet.
Anschließend betrachten wir zwei verbreitete Anwendungen der Singulärwertzerlegung ausführlicher:
die \emph{Hauptkomponentenanalyse}, ein statistisches Verfahren der Data Science, welches komplexe Daten auf ihr Wesentliches reduziert sowie die bereits angesprochenen Empfehlungssysteme, bei denen unter anderem untersucht wird, wie eine Matrixzerlegung überhaupt mit Nutzer-Film-Bewertungen zusammenhängt.
Den Abschluss des Hauptteils bildet die Implementierung eines eigenen Empfehlungssystems in \texttt{Python}, basierend auf realen Nutzerbewertungen aus einer Online-Datenbank.

Es sei darauf hingewiesen, dass grundlegende Kenntnisse in \texttt{Python} zum Verständnis der Implementierung vorausgesetzt werden.
Innerhalb des Textes werden zentrale Passagen des Codes erläutert, während der vollständige Programmcode --- mit Kommentaren an den nicht besprochenen Stellen --- jeweils mit Verweis im Anhang zu finden ist.
